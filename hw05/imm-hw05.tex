\documentclass[12pt]{jhwhw}
\author{Ian Malerich}
\title{Com S 352: Homework 5}
\usepackage{amssymb, amsfonts, mathtools, graphicx, breqn}
\usepackage{minted, subfig, float, scrextend, setspace, soul, amsthm, multirow}
\usemintedstyle{friendly}

\onehalfspacing
\begin{document}
\raggedright

%% Problem 7.15
\textbf{7.15}  
	Compare the circular-wait scheme with the various deadlock-avoidance schemes 
	(like the banker's algorithm with respect to the following issues:
	\begin{enumerate}
		\item Runtime overheads
		\item System throughput
	\end{enumerate}
\textcolor[RGB]{240,240,240}{\rule{\textwidth}{0.5pt}}\bigbreak

	\begin{addmargin}[1em]{}
	\end{addmargin}

%% Problem 7.16
\clearpage
\textbf{7.16}  
	In a real computer system, neither the resources available nor the demands of processes for resources are 
	consistent over long periods (months). Resources break or are replaced, new processes come and go, and new 
	resources are bought and added to the system. If deadlock is controlled by the banker's algorithm,
	which of the following changes can be made safely (without introducing the possibility of deadlock),
	and under what circumastances?
	\begin{enumerate}
		\item Increase \textbf{Available}. \\
			This can pretty much always be made, if the system is in a safe state, then there
			is sufficient resources for all processes introducing more resources does not change this.
			The sequence of process which justifies the safe state of the system will continue
			to justify the safe status with more available resources.
			Note that by introducing more resources, there may be more possible ordering of the processes
			which maintain a safe state.
		\item Decrease \textbf{Available}. \\
			This is much more risky, if resources are removed, the system will need to be reevaluated,
			and it is possible that the system will enter a deadlock. For example, at an extreme, if 
			at least one process needs resources, and you remove all resources, you have now introduced deadlock.
			There are situations where decreasing available resources may be okay, for example
			if you have a safe state which you add $n$ (of the same) resources too, and then remove $n$, you are back
			where you started which was initial safe, thus you are still safe.
		\item Increase \textbf{Max} for one process. \\
			This could introduce deadlock. The trivial example is if you increase resources beyond what is available.
			Another example might be if this process is the only process that could run first in the safe sequence,
			but by increasing its maximum required resources, you could introduce a situation where it no longer
			fits that role, no process can run first, thus no such sequence exists anymore, therefore
			the system is no longer in a safe state and deadlock is possible.
		\item Decrease \textbf{Max} for one process. \\
			This will always be fine, if the system is in a safe state, having a process require less
			of a particular resource is never a bad thing. By doing this, more possible orderings
			of processes may maintain a safe state, in particular, the process which requires less 
			maximum resources may be able to be executed earlier in the sequence than it was before
			decreasing maximum resources.
	\end{enumerate}

%% Problem 7.18
\clearpage
\textbf{7.18}  
	Consider a system consisting of $m$ resources of the same type being shared by $n$ processes. A
	process can request or release only one resource at a time. Show that the system is deadlock free
	if the following two conditions hold.
	\begin{enumerate}
		\item The maximum need of each process is between one resource and $m$ resources.
		\item The sum of all maximum needs is less than $m+n$.
	\end{enumerate}
\textcolor[RGB]{240,240,240}{\rule{\textwidth}{0.5pt}}\bigbreak

	\begin{addmargin}[1em]{}
	\end{addmargin}

%% Problem 7.22
\clearpage
\textbf{7.22}  
	Consider the following snapshot of a system: \bigbreak
	\begin{tabular}{|l||c|c|c|c||c|c|c|c||c|c|c|c|}
		\hline
		& \multicolumn{4}{|c||}{Allocation} & \multicolumn{4}{|c||}{Max} & \multicolumn{4}{|c|}{Need} \\ \hline
		& A & B & C & D & A & B & C & D & A & B & C & D\\ \hline
		$P_0$ & 3 & 0 & 1 & 4 & 5 & 1 & 1 & 7 & 2 & 1 & 0 & 3 \\ \hline
		$P_1$ & 2 & 2 & 1 & 0 & 3 & 2 & 1 & 1 & 1 & 0 & 0 & 1 \\ \hline
		$P_2$ & 3 & 1 & 2 & 1 & 3 & 3 & 2 & 1 & 0 & 2 & 0 & 0 \\ \hline
		$P_3$ & 0 & 5 & 1 & 0 & 4 & 6 & 1 & 2 & 4 & 1 & 0 & 2 \\ \hline
		$P_4$ & 4 & 2 & 1 & 2 & 6 & 3 & 2 & 5 & 2 & 1 & 1 & 3 \\ \hline\hline
		$Total$ & 12 & 10 & 6 & 7 & & & & & & & & \\ \hline
	\end{tabular}

	\bigbreak
	Using the banker's algorithm, determine whether or not each of the following states is unsafe.
	If the state is safe, illustrate the order in which the processes may complete. Otherwise, illustrate why 
	the state is unsafe. \\
	\begin{enumerate}
		\item \textbf{Available} = (0,3,0,1) \\
			Note that total memory = $(12+0, 10+3, 6+0, 7+1) = (12, 13, 6, 8)$ \\
			$P_2 \rightarrow (3,4,2,2) \rightarrow P_4 \rightarrow (7, 6, 3, 4)
			\rightarrow P_0 \rightarrow (10, 6, 4, 8) \rightarrow P_1 \rightarrow (12, 8, 5, 8)
			\rightarrow P_3 (12, 13, 6, 8)
			$. \\
			Thus we can see from the above sequence that the system is in a safe state according
			to the banker's algorithm.
		\item \textbf{Available} = (1,0,0,2) \\
			Note that total memory = $(12+1, 10+00, 6+0, 7+2) = (13, 10, 6, 9)$ \\
			Initially, we can only run $P_1 \rightarrow (3, 2, 1, 2)$, next we must
			run $P_2 \rightarrow (6, 3, 3, 3)$ then things start getting much easier,
			as we have enough resources to run any process. $P_3 \rightarrow (6, 8, 4, 3)
			\rightarrow P_4 \rightarrow (10, 10, 5, 5) \rightarrow P_0 \rightarrow (13, 10, 6, 9)$. \\
			Thus again we see that following the above sequence we are able to run every process, and are
			therefore in a safe state according to the banker's algorithm.
	\end{enumerate}
\bigbreak

%% Problem 7.22
\clearpage
\textbf{7.22}  
	What is the optimistic assumption made in the deadlock-detection algorithm?
	How can this assumption be violated?
\textcolor[RGB]{240,240,240}{\rule{\textwidth}{0.5pt}}\bigbreak

	\begin{addmargin}[1em]{}
		The assumption is that once all tasks are in a safe state, that they will make no
		further requests will be made. Doing so would violate the assumption and could lead 
		to an unsafe state, and thus deadlock.
	\end{addmargin}

\end{document}
