\documentclass[12pt]{jhwhw}
\author{Ian Malerich}
\title{Com S 352: Homework 4}
\usepackage{amssymb, amsfonts, mathtools, graphicx, breqn}
\usepackage{minted, subfig, float, scrextend, setspace, soul, amsthm}
\usemintedstyle{friendly}

\onehalfspacing
\begin{document}
\raggedright

%% Problem 5.8
\textbf{5.8}  
	The first known correct software solution to the critical-section problem
	for two processes was developed by Dekker. The two processes, P0 and P1,
	shared the following variables: \\
	\begin{minted}{c}
	boolean flag[2]; /* initially false */
	int turn;
	\end{minted}
	The structure of process $P_i$ ($i\in \{0,1\}$) is shown in Figure 5.21. The other 
	process is $P_j$ ($j\in \{0,1\}$). Prove that the algorithm satisfies all three
	requirements for the critical-section problem.

	\inputminted[linenos]{c}{5.8.c}
\textcolor[RGB]{240,240,240}{\rule{\textwidth}{0.5pt}}\bigbreak

	\clearpage
	\begin{addmargin}[1em]{}
		\begin{proof}\ \\
			\bigbreak
			1. Mutual Exclusion \\
			Each process sets their flag to true \hl{[2]}, indicating that they want access
			to the mutual data, this  flag will be set to false once they have
			completed their critical section \hl{[16]}.
			They then both enter a loop which cannot be broken out of while
			the other process wants access to the critical section \hl{[4]}.
			Turn must initially be $i$ or $j$, if it is the other processes turn,
			the current process will give up its flag \hl{[6]}, 
			breaking the other out of the loop then
			the current process will in turn enter a new loop waiting for turn to change
			(to itself) hl{[8]}.
			The other process can now execute it's critical section, set the turn to break
			the other process out of the inner loop \hl{[8]}(which sets the other flag to true), 
			and then the current process turns off flag breaking the other
			process out of the outer loop \hl{[4]}. 
			The other process can now execute critical section code in turn.
			If the current process after executing its critical section reenters the loop 
			and sets flag back to true before the other process is scheduled \hl{[2]}, the other
			process will continue to loop \hl{[4]}, 
			However as turn has changed, the current process will give up its flag at
			\hl{[6]} which finally allows the other process, and only the other process
			to enter the critical section. Repeating the logic above.

			\bigbreak
			2. Progress \\
			If no process is currently executing the critical section, as turn is set,
			one process must be ready to enter the critical section. If the process who's turn
			does not wish to enter the critical section (i.e. they are in the remainder section)
			then they must have set turn to the other process \hl{[15]} thus guaranteeing 
			the other process (who wants access) entrance into the critical section.
			If both processes do not want access to the critical section, then one
			must be in their remainder section while the first is still trapped
			in the loop, again, as turn has changed \hl{[15]} one process will break
			out of the loop at \hl{[8]} switching its state to desire access to the critical
			section, which it will then be free to execute.

			\bigbreak
			3. Bounded Waiting \\
			Preserved via the turn variable. As mentioned above, the turn variable
			switches state after each execution of the critical section. This guarantees
			that the other process will be the next to enter its critical section.
		\end{proof}
	\end{addmargin}
	\clearpage

%% Problem 5.11
\textbf{5.11}  
	Explain why interrupts are not appropriate for implementing synchronization
	primitives in multiprocessor systems.
\textcolor[RGB]{240,240,240}{\rule{\textwidth}{0.5pt}}\bigbreak

	\begin{addmargin}[1em]{}
		If a process does not wish to give up the its processor via interrupts, it
		can only guarantee mutual exclusion for processes which would run on the
		same processor. Since we are assuming a multiprocessor system, it is possible
		that another process could modify the data which we are trying to guarantee mutual
		exclusion on, by simply running on another of the systems processors.
		Thus this solution is insufficient.
	\end{addmargin}
	\bigbreak

%% Problem 5.16
\textbf{5.16}  
	The implementation of mutex locks provided in Section 5.5 suffers
	from busy waiting. Describe what changes would be necessary so that a process 
	waiting to acquire a mutex lock would be blocked and placed in a waiting queue
	until the lock became available.
\textcolor[RGB]{240,240,240}{\rule{\textwidth}{0.5pt}}\bigbreak

	\begin{addmargin}[1em]{}
		We would need to introduce a queue where processes can sleep (not execute)
		while they are waiting for a locked mutex. They will enter this queue
		on a call to acquire for a mutex which is unavailable, and will be released
		from this queue, when the process who currently holds the lock calls release.
		Note I am using the books terminology from Section 5.5, the lecture equivalent
		would be wait() and signal() respectively.
		Below, we can see a rough implementation modifying the books implementation,
		which incorporates queue's in place of the while loop.
		Note that like the book, I am not using testAndSet with a lock to guarantee
		only one process modifies available at a time, thus available is assumed to 
		be thread safe.

		\inputminted{c}{5.16.c}
	\end{addmargin}
	\bigbreak

%% Problem 5.29
\textbf{5.29}  
	How does the signal() operation with monitors differ from the corresponding
	operation defined for semaphores?
\textcolor[RGB]{240,240,240}{\rule{\textwidth}{0.5pt}}\bigbreak

	\begin{addmargin}[1em]{}
		If a signal is performed on a monitor with no waiting queue, the signal is
		simply ignored. If a wait is then performed, the calling thread will simply
		be blocked. With a semaphore, the signal call will never be ignored as it will
		always at least increment the semaphore value. A future wait call will, instead
		of blocking, immediately succeed due to this incremented value.
	\end{addmargin}
	\bigbreak

%% Problem 5.32
\textbf{5.32}  
	A file is to be shared among different processes, each of which has a unique number.
	The file can be accessed simultaneously by several processes, subject to the following
	constraint: the sum of all unique numbers associated with all the processes accessing
	the file must be less than $n$. Write a monitor to coordinate access to the file.
\textcolor[RGB]{240,240,240}{\rule{\textwidth}{0.5pt}}\bigbreak

	\begin{addmargin}[1em]{}
		The monitor is fairly straightforward, assumes $n$ is a constant defined elsewhere,
		further assumes $pnum$ is the corresponding process number to a process which
		is also available from elsewhere. Note, there is the possibility of a blocking
		process, if $pnum > n$ then even if no processes are accessing the file, this process
		will never leave the queue as there is not enough room to access the file.
		Further, a call to signal with a non-empty queue does not guarantee a 
		process will access the file, even if there is a process in the queue with a small
		enough pnum to read the file, the queue will continue to wait until the head has
		enough room (this is why a while loop is used instead of an if statement).
		\inputminted{c}{5.32.c}
	\end{addmargin}
	\bigbreak

\end{document}
