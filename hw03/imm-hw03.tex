\documentclass[12pt]{jhwhw}
\author{Ian Malerich}
\title{Com S 352: Homework 3}
\usepackage{amssymb, amsfonts, mathtools, graphicx, breqn}
\usepackage{minted, subfig, float, scrextend, setspace, soul}
\usemintedstyle{friendly}

\onehalfspacing
\begin{document}
\raggedright

%% Problem 4.8
\textbf{4.8 (8 points)}  Which of the following components of program state are 
	shared across threads in a multithreaded process.
	\begin{enumerate}
		\item Register Values
		\item Heap Memory
		\item Global Variables
		\item Stack Memory
	\end{enumerate}
\textcolor[RGB]{240,240,240}{\rule{\textwidth}{0.5pt}}\bigbreak

	\begin{addmargin}[1em]{}
	\end{addmargin}

%% Problem 4.11
\textbf{4.11 (5 points)} Is it possible to have concurrency but not parallelism?
	Explain.
\textcolor[RGB]{240,240,240}{\rule{\textwidth}{0.5pt}}\bigbreak

	\begin{addmargin}[1em]{}
		Yes. Parallelism requires multiple CPU's, so that multiple processes/threads
		are being run simultaneously in real time. Concurrency only requires that more
		than one process/thread is in the process of being computed at a time, thus,
		multiple process/thread may share a single CPU (not parallel) and alternate
		turns executing code by following some scheduling algorithm.
	\end{addmargin}
	\bigbreak

%% Problem 4.17
\textbf{4.17 (10 points)} The program shown below uses the pthreads API. 
	What would be the output from the program at LINE C and LINE P?
\inputminted{c}{4.17.c}
\textcolor[RGB]{240,240,240}{\rule{\textwidth}{0.5pt}}\bigbreak

	\begin{addmargin}[1em]{}
		C = 5 \\
		P = 0 \\
	\end{addmargin}

%% Problem 6.14
\textbf{6.14 (15 points)} Consider the exponential average formula used to
	predict the length of the next CPU burst. What are the implications of assigning
	the values to the parameters used by the algorithm?
	\begin{enumerate}
		\item $\alpha=0$ and $\tau_0=100$ ms
		\item $\alpha=0.99$ and $\tau_ 0=10$ ms
	\end{enumerate}
\textcolor[RGB]{240,240,240}{\rule{\textwidth}{0.5pt}}\bigbreak

	\begin{addmargin}[1em]{}
	\end{addmargin}

%% Problem 6.16
\textbf{6.16 (30 points)} 
	Consider the following set of processes, with the length of the CPU
	burst given in milliseconds:
	Process Burst Time Priority \\
	P1 2 2 \\
	P2 1 1 \\
	P3 8 4 \\
	P4 4 2 \\
	P5 5 3 \\
	The processes are assumed to have arrived in the order P1, P2, P3, P4, P5, all at 
	time 0.
	\begin{enumerate}
		\item Draw foru Gantt charts that illustrate the execution of these processes using the
			following scheduling algorithms: FCFS, SJF, non-preemptive priority (a larger
			priority number implies a higher priority), and RR (quantum = 2).
		\item What is the turnaround time of each process for each of the scheduling algorithms
			in part a?
		\item What is the waiting time of each process for each of these scheduling algorithms.
		\item Which of the algorithms results in the minimum average waiting time
			(over all processes)?
	\end{enumerate}
\textcolor[RGB]{240,240,240}{\rule{\textwidth}{0.5pt}}\bigbreak

	\begin{addmargin}[1em]{}
	\end{addmargin}

%% Problem 6.19
\textbf{6.19 (5 points)} 
	Which of the following scheduling algorithms could result in starvation?
	\begin{enumerate}
		\item First-Come, First-Served
		\item Shortest job first
		\item Round robin
		\item Priority
	\end{enumerate}
\textcolor[RGB]{240,240,240}{\rule{\textwidth}{0.5pt}}\bigbreak

	\begin{addmargin}[1em]{}
	\end{addmargin}

%% Problem 6.23
\textbf{6.23 (7 points)} 
	Consider a preemptive priority scheduling algorithm based on dynamically
	changing priorities. Larger priority numbers imply higher priority.
	When a process is waiting for the CPU (in the ready queue, but not
	running), its priority change at a rate $\alpha$. When it is running, its
	priority changes at a rate $\beta$. All processes are given a priority of 0 when they
	enter the ready queue. The parameters $\alpha$ and $\beta$ can be set to give 
	many different scheduling algorithms.
	\begin{enumerate}
		\item What is the algorithm that results from $\beta > \alpha > 0$?
		\item What is the algorithm that results from $\alpha < \beta < 0$?
	\end{enumerate}
\textcolor[RGB]{240,240,240}{\rule{\textwidth}{0.5pt}}\bigbreak

	\begin{addmargin}[1em]{}
	\end{addmargin}

%% Problem 7
\textbf{7 (20 points)} Convert the following program to use threads.
	Under the following restrictions:
	\begin{enumerate}
		\item One thread will print ``hello", one thread will print ``world", and the
			main function will print the trailing $``\backslash n"$, using just
			pthread_create(), pthread_exit(), pthread_yield(), and
			pthread_join().
		\item You must use a synchronization method to ensure the "world" thread
			runs after the "hello" thread.
		\item You must use a synchronization method to ensure that the main
			thread does not execute until after the "world" thread.
	\end{enumerate}

\textcolor[RGB]{240,240,240}{\rule{\textwidth}{0.5pt}}\bigbreak
\inputminted{c}{7.c}

\end{document}
