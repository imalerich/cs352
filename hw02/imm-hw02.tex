\documentclass[12pt]{jhwhw}
\author{Ian Malerich}
\title{Com S 352: Homework 1}
\usepackage{amssymb, amsfonts, mathtools, graphicx, breqn}
\usepackage{minted, subfig, float, scrextend, setspace, soul}
\usemintedstyle{friendly}

\onehalfspacing
\begin{document}
\raggedright

%% Problem 3.8
\textbf{3.8 (15 points)} Describe the difference between short-term, medium-term, and 
	long-term scheduling.
\textcolor[RGB]{240,240,240}{\rule{\textwidth}{0.5pt}}\bigbreak

	\begin{addmargin}[1em]{}
		Answers look like this.
	\end{addmargin}

%% Problem 3.9
\bigbreak
\textbf{3.9 (10 points)} Describe the actions taken by a kernel to context-switch
	between processes.
\textcolor[RGB]{240,240,240}{\rule{\textwidth}{0.5pt}}\bigbreak

	\begin{addmargin}[1em]{}
		Answers look like this.
	\end{addmargin}

%% Problem 3.12
\bigbreak
\textbf{3.12 (25 points)} Including the initial process, how many processes are created by 
	the program shown below.
\inputminted{c}{3.12.c}
\textcolor[RGB]{240,240,240}{\rule{\textwidth}{0.5pt}}\bigbreak

	\begin{addmargin}[1em]{}
		\hl{16 processes are created by the program.}
		\bigbreak
		To see this, note that the parent creates 4 child processes (one at each
		index of the for loop). Thus, we so far have \hl{5 processes total}, 
		including the parent. \\
		Consider the process created at index 0, when this process starts, it will run
		the remaining 3 iterations of the for loop, and create 3 child processes of its
		own. The first of those will create 2 (one of which will create another 1), the second
		will create 1 child process, and the last will terminate the loop when it is created
		and not make another child. Thus the child at index 0 has \hl{7 children/grandchildren.} \\
		The second child (index 1) process will create 2 child process for each iteration, one of
		which will create 1 more and the last will again terminate the loop, 
		for a \hl{total of 3.} \\
		The third child (index 2) will create 1 child process, with no children, 
		for a total of \hl{1 children/grandchildren.} \\
		The fourth and final child (index 3) immediately terminates the lop, 
		for a \hl{total of 0 children/grandchildren}.
		\\
		Thus, 5 + 7 + 3 + 1 + 0 = 16 processes in total.
	\end{addmargin}

%% Problem 3.14
\bigbreak
\textbf{3.14 (25 points)} Using the program below, identify the values of pid at
	lines A, B, C, and D (Assume that the actual pids of the parent and child
	are 2600 and 2603, respectively).
\inputminted{c}{3.14.c}
\textcolor[RGB]{240,240,240}{\rule{\textwidth}{0.5pt}}\bigbreak

	\begin{addmargin}[1em]{}
		\textbf{A:} 0 &(pid = fork() returns 0 to child process) \\
		\textbf{B:} 2603 &(pid1 = getpid() returns id of current process, i.e. the child) \\
		\textbf{C:} 2603 &(pid = fork() returns child pid to the parent) \\
		\textbf{D:} 2600 &(pid1 = getpid() returns id of current process, i.e. the parent)\\
	\end{addmargin}

%% Problem 3.17
\bigbreak
\textbf{3.17 (25 points)} Using the program shown below, explain what the output will be
	at lines X and Y.
\inputminted{c}{3.17.c}
\textcolor[RGB]{240,240,240}{\rule{\textwidth}{0.5pt}}\bigbreak

	\begin{addmargin}[1em]{}
		\textbf{LINE X:} The child process multiplies each value of the array by the opposite
		of the values index. Thus the child will print out the results of
		\{0*-0, 1*-1, 2*-2, 3*-3, 4*-4\}. Which will be the values
		\{0, -1, -4, -9, -16\}.
		\bigbreak
		\hl{CHILD: 0 CHILD: -1 CHILD: -4 CHILD: -9 CHILD: -16}

		\textbf{LINE Y:} The child modifies the values of the array in it's own memory space
		(even though it is global, child process still have their own global memory space),
		not the parents, thus the parent will print out the results the array was
		initialized to when 'nums' was declared. That is, \{0, 1, 2, 3, 4\}.
		\bigbreak
		\hl{PARENT: 0 PARENT: 1 PARENT: 2 PARENT: 3 PARENT: 4}
	\end{addmargin}

\end{document}
