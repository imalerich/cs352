\documentclass[12pt]{jhwhw}
\author{Ian Malerich}
\title{Com S 352: Homework 6}
\usepackage{amssymb, amsfonts, mathtools, graphicx, breqn}
\usepackage{minted, subfig, float, scrextend, setspace, soul, amsthm, multirow}
\usemintedstyle{friendly}

\onehalfspacing
\begin{document}
\raggedright

\textbf{8.13}  
	Compare the memory organization schemes of contiguous memory allocation, 
	pure segmentation, and pure paging with respect to the following issues.
	\begin{enumerate}
		\item External Fragmentation
		\item Internal Fragmentation
	\end{enumerate}
\textcolor[RGB]{240,240,240}{\rule{\textwidth}{0.5pt}}\bigbreak

	\begin{addmargin}[1em]{}
		With \hl{contiguous memory} as new processes are allocated and released, because they are tightly
		packed, processes leaving memory will leave gaps which may or may not be filled, thus contiguous
		memory suffers from external fragmentation.

		\bigbreak
		\hl{Pure paging} will similarly suffer from external fragmentation as a process segment is still
		aligned continuously in memory. Segments of processes which have been released my be replaced
		by smaller processes, still leaving holes.

		\bigbreak
		By using \hl{pure paging} external fragmentation can be removed as each processes is asigned
		discrete well defined, equal sized blocks of memory. Thus the system can guarantee that
		each block of memory is assigned to some process (if need be). However, pure paging will
		now suffer from internal fragmentation, pages not completely utilized by a process
		lead to a waste of space (internal fragmentation).
	\end{addmargin}

\bigbreak
\textbf{8.20}  
	Assuming a 1-KB page size, what are the page numbers for the following address
	references (provided as decimal numbers):
	\begin{enumerate}
		\item 3085 
			$$ \frac{3085}{1024} = 3.012695 \Rightarrow 3$$
		\item 42095
			$$ \frac{42095}{1024} = 41.1084 \Rightarrow 41$$
		\item 215201
			$$ \frac{215201}{1024} = 210.1572 \Rightarrow 210$$
		\item 650000
			$$ \frac{650000}{1024} = 634.7656 \Rightarrow 634$$
		\item 2000001
			$$ \frac{2000001}{1024} = 1953.126 \Rightarrow 1953$$
	\end{enumerate}
\textcolor[RGB]{240,240,240}{\rule{\textwidth}{0.5pt}}\bigbreak

\textbf{8.23}  
	Consider a logical address space of 256 pages with a 4-KB page size, mapped onto a physical
	memory of 64 frames.
	\begin{enumerate}
		\item How many bits are required in the logical address? \\
			$$256\times 4\times 1024 = 1048576\text{ logical memory locations }
			\Rightarrow log_2(1048576) = 20\text{ bits }$$
		\item How many bits are required in the physical address?
			$$64\times 4\times 1024 = 262144\text{ physical memory locations }
			\Rightarrow log_2(262144) = 18\text{ bits }$$
	\end{enumerate}
\textcolor[RGB]{240,240,240}{\rule{\textwidth}{0.5pt}}\bigbreak

\textbf{9.21}  
	Consider the following page reference string: \\
	$$ 7,2,3,1,2,5,3,4,6,7,7,1,0,5,4,6,2,3,0,1 $$
	Assuming demand paging with three frames, how many page faults would occur for the following
	replacement algorithms?
	\begin{enumerate}
		\item LRU Replacement \\
			\begin{addmargin}[-3em]{}
			\begin{tabular}{|c|c|c|c|c|c|c|c|c|c|c|c|c|c|c|c|c|c|c|c|}
				\hline
				$7_m$ & $2_m$ & $3_m$ & $1_m$ & $2_h$ & $5_m$ & 
				$3_m$ & $4_m$ & $6_m$ & $7_m$ & $7_h$ & $1_m$ & 
				$0_m$ & $5_m$ & $4_m$ & $6_m$ & $2_m$ & $3_m$ & $0_m$ & $1_m$ \\
				\hline\hline
				7&7&7&1&1&1&3&3&3&7&7&7&7&5&5&5&2&2&2&1\\ \hline
				 &2&2&2&2&2&2&4&4&4&4&1&1&1&4&4&4&3&3&3\\ \hline
				 & &3&3&3&5&5&5&6&6&6&6&0&0&0&6&6&6&0&0\\ \hline
			\end{tabular}
			\end{addmargin}
			\hl{18 page faults.}
		\item FIFO Replacement \\
			\begin{addmargin}[-3em]{}
			\begin{tabular}{|c|c|c|c|c|c|c|c|c|c|c|c|c|c|c|c|c|c|c|c|}
				\hline
				$7_m$ & $2_m$ & $3_m$ & $1_m$ & $2_h$ & $5_m$ & 
				$3_h$ & $4_m$ & $6_m$ & $7_m$ & $7_h$ & $1_m$ & 
				$0_m$ & $5_m$ & $4_m$ & $6_m$ & $2_m$ & $3_m$ & $0_m$ & $1_m$ \\
				\hline\hline
				 7&7&7&1&1&1&1&1&6&6&6&6&0&0&0&6&2&2&2&1\\ \hline
				  &2&2&2&2&5&5&5&5&7&7&7&7&5&5&5&5&3&3&3\\ \hline
				  & &3&3&3&3&3&4&4&4&4&1&1&1&4&4&4&4&0&0\\ \hline
			\end{tabular}
			\end{addmargin}
			\hl{17 page faults.}
		\item Optimal Replacement \\
			\begin{addmargin}[-3em]{}
			\begin{tabular}{|c|c|c|c|c|c|c|c|c|c|c|c|c|c|c|c|c|c|c|c|}
				\hline
				$7_m$ & $2_m$ & $3_m$ & $1_m$ & $2_h$ & $5_m$ & 
				$3_h$ & $4_m$ & $6_m$ & $7_m$ & $7_h$ & $1_h$ & 
				$0_m$ & $5_h$ & $4_m$ & $6_m$ & $2_m$ & $3_m$ & $0_h$ & $1_h$ \\
				\hline\hline
				7&7&7&1&1&1&1&1&1&1&1&1&1&1&1&1&1&1&1&1\\ \hline
				 &2&2&2&2&5&5&5&5&5&5&5&5&5&4&6&2&3&3&3\\ \hline
				 & &3&3&3&3&3&3&6&7&7&7&0&0&0&0&0&0&0&0\\ \hline
			\end{tabular}
			\end{addmargin}
			\hl{13 page faults.}
	\end{enumerate}
\textcolor[RGB]{240,240,240}{\rule{\textwidth}{0.5pt}}\bigbreak

	\begin{addmargin}[1em]{}
	\end{addmargin}

\textbf{9.27}  
	Consider a demand-paging system with the following time-measured utilizations: \\
	CPU utilization 20\% \\
	Paging disk 97.7\% \\
	Other I/O devices 5\% \\
	For each of the following, indicate whether it will (or is likely to) improve
	CPU utilization. Explain your answers.
	\begin{enumerate}
		\item Install a faster CPU. \\
			This will obviously not improve CPU utilization, as it stands, we are currently
			only using 20\%, our current CPU has plenty of room for growth, having faster
			processing will only decrease current utilization.
		\item Install a bigger paging disk. \\
			Could help, but not likely. Our high paging usage is due to insufficient main
			memory such that our processes need to frequently access the paging disk.
			Increasing the paging disk does not help solve this main problem.
		\item Increase the degree of multiprogramming. \\
			This will not help. Increasing the degree of multiprogramming will require us to 
			keep more pages in memory, but clearly as our utilization is 97.7\% we cannot
			afford this, our other processes will suffer and hurt our CPU utilization.
		\item Decrease the degree of multiprogramming. \\
			On the contrary, this will indeed help. Each process can keep pages around in
			memory and will decrease paging disk utilization. Less time will be spent
			by processes waiting for the paging disk and processes will then run more efficiently.
		\clearpage
		\item Install more main memory. \\
			This will probably help the most, our biggest bottle neck is waiting for the paging
			disk. With more memory, the paging disk will need to be used much less,
			therefore processes can continue executing longer increasing CPU utilization.
		\item Install a faster hard disk or multiple controllers with multiple hard disks. \\
			We are only using 5\% I/O utilization, improving the speed of I/O operations is
			not likely to help much as noted before our problem is with paging not I/O.
		\item Add pre paging to the page-fetch algorithms. \\
			This could help, paging disk utilization is already very high, and pre paging
			does not come for free so this is not likely practical. At best this will allow
			processes to run sooner and for longer and may slightly bump CPU utilization.
		\item Increase the page size. \\
			Again, this might help. But decreasing the degree of multiprogramming or installing
			more main memory would be a much better solution. The best this could do is allow
			a current running process to run for a little bit longer before it page faults.
			However paging disk operations will take longer, so these may cancel out the benefit.
	\end{enumerate}
\textcolor[RGB]{240,240,240}{\rule{\textwidth}{0.5pt}}\bigbreak

\textbf{9.32}  
	What is the cause of thrashing? How does the system detect thrashing? Once it detects 
	thrashing, what can the system do to eliminate this problem?
\textcolor[RGB]{240,240,240}{\rule{\textwidth}{0.5pt}}\bigbreak

	\begin{addmargin}[1em]{}
		If a system does not meet the minimum number of pages required by a process, pages will
		continuously fault causing disk operations, which in turn cause context switches, which cause
		further page faults taking up necessary pages for the original process. This cycle of page faults
		leading to no net work is known as thrashing. \\
		To eliminate this problem, the system can detect the thrashing by analyzing and evaluating
		CPU utilization relative to degree of multiprogramming. If thrashing occurs, the system
		can decrease the level of multiprogramming, allowing processes time to store all necessary
		pages in memory to perform their work (without a context switch interrupting and disrupting them).
	\end{addmargin}

\bigbreak
\textbf{9.34}  
	Consider the parameter $\Delta$ used to define the working-set window in the working-set model.
	When $\Delta$ is set to a small value, what is the effect on the page-fault frequency and the number
	of active (non-suspended) processes currently executing in the system? What is the effect when $\Delta$
	is set to a very high value?
\textcolor[RGB]{240,240,240}{\rule{\textwidth}{0.5pt}}\bigbreak

	\begin{addmargin}[1em]{}
		With a small $\Delta$ value, required pages for a processes may not be met, causing
		processes to run which do not have the resources to run. This results in a large
		number of page faults, with many active processes.
		\bigbreak
		Conversely, with a large $\Delta$ value, required pages for a process will be overestimated,
		this will decrease the current number of working processes, but each process
		will have the resources to run. Thus there will be a small number of 
		page faults, and fewer processes, but due to the overestimating processes may not be scheduled
		even if there are enough resources available.
	\end{addmargin}

\end{document}
