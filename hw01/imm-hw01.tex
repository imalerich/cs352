\documentclass[12pt]{jhwhw}
\author{Ian Malerich}
\title{Com S 352: Homework 1}
\usepackage{amssymb, amsfonts, mathtools, graphicx, breqn}
\usepackage{minted, subfig, float, scrextend, setspace}
\usemintedstyle{friendly}

\onehalfspacing
\begin{document}
\raggedright

%% Problem 1.14
\textbf{1.14 (10 points)} Under what circumstances would a user be better off using
a timesharing system rather than a PC or a single-user workstation.
\textcolor[RGB]{240,240,240}{\rule{\textwidth}{0.5pt}}\bigbreak

	\begin{addmargin}[1em]{}
		Timesharing can be used where a single-user workstation is not financially or
		practically (i.e. available space) possible. As a consequence of those issues, a facility
		may only have one or two computers available, but a large number of users
		who need to perform some sort of work on the machine. Rather than taking turns
		(single-user) on those machines, time sharing can be used to allow all users
		to use the one system simultaneously. \\
		The system can periodically switch between working each users jobs, thus creating
		the illusion that each user is using their own personal system, when in fact
		there is only one machine.
	\end{addmargin}

%% Problem 1.20
\bigbreak
\textbf{1.20 (10 points)} Direct memory access is used for high-speed I/O devices
in order to avoid increasing the CPU's execution load.
\begin{enumerate}
	\item How does the CPU interface with the device to coordinate the transfer?
	\item How does the CPU know when the memory operations are complete?
	\item The CPU is allowed to execute other programs while the DMA controller is
		transferring data. Does this process interfere with the execution of the user programs?
		If so, describe what forms of interference are caused.
\end{enumerate}
\textcolor[RGB]{240,240,240}{\rule{\textwidth}{0.5pt}}\bigbreak

	\begin{addmargin}[1em]{}
		\textbf{(a)} The operating system uses device drivers (which run on the kernel) to
			communicate with the I/O devices. In order to allow the CPU to continuing 
			executing programs during I/O operations, a DMA can be used, this can load 
			device drivers, wait for device operations to complete, and then send
			an interrupt to the CPU once the operation is complete.
		\bigbreak
		\textbf{(b)} The I/O device can send an interrupt to the CPU, letting it know that
			the memory operations are complete. The CPU now knows that that memory location is 
			once again available, and can resume the process requiring the completion
			of that I/O operation at the CPU's leisure.
		\bigbreak
		\textbf{(c)} In general, no, the CPU can run any other process freely while the DMA
			controller is transferring data. There may be small situations where
			for example if the DMA is busy another program may have to wait to make memory
			requests. Further when the DMA sends an interrupt, this can cause the current
			user program being executed by the CPU to be temporarily suspended.
	\end{addmargin}

%% Problem 1.25
\bigbreak
\textbf{1.25 (10 points)} Describe a mechanism enforcing memory protection in order to prevent
a program from modifying the memory associated with other programs.
\textcolor[RGB]{240,240,240}{\rule{\textwidth}{0.5pt}}\bigbreak

	\begin{addmargin}[1em]{}
		First we note that user programs run in different modes (user mode vs. kernel mode).
		The CPU enforces that memory operations can ONLY be performed in kernel mode.
		Next, user programs are given their own 'virtual' memory spaces which are
		independent of physical device locations (these are known only to the kernel).
		Since memory operations must go through the kernel, the kernel can easily check
		that addresses given are in fact within the local 'virtual' memory space assigned
		to the program, if so the location can be converted to a physical address and memory
		operations performed without ever telling the program the physical address. However,
		if the address is not within the local memory space of the user program, an exception
		can be generated and sent to the user program (segmentation fault) causing the
		system call to fail, thus protecting memory used by other programs.
	\end{addmargin}

%% Problem 2.18
\bigbreak
\textbf{2.18 (10 points)} What are the two models of inter-process communication? What
are the strengths and weaknesses of the two approaches.
\textcolor[RGB]{240,240,240}{\rule{\textwidth}{0.5pt}}\bigbreak

	\begin{addmargin}[1em]{}
		\textbf{Message Passing Model:} With a message passing model, messages can be sent
			in either a blocking or non-blocking way (synchronous or asynchronous). This
			method is much easier to implement, but isn't very suitable for large 
			amounts of data as message passing requires more overhead (as communication
			is treated as a file, whether it be physical (FIFO) or not (pipe)) and thus
			results in slower speeds.
		\bigbreak
		\textbf{Memory Sharing Model:} Memory is allocated which both processes have access
			to. Thus read/write operations are direct, and thus much faster than message
			passing, making memory sharing much more suitable for large blocks of data.
			However, memory sharing can cause synchronization issues when multiple processes
			read/write to the same location. A simple example is if both processes get a value
			and then increment it, depending on the order of the 2 sets of get/set operations, the
			value may be incremented by 1 or 2 
			(\{get1,get2,set1,set2\}$\Rightarrow+1$ vs \{get1,set1,get2,set2\}$\Rightarrow+2$)!
	\end{addmargin}

%% Problem 2.23
\clearpage
\textbf{2.23 (10 points)} How are iOS and Android similar? How are they different?
\textcolor[RGB]{240,240,240}{\rule{\textwidth}{0.5pt}}\bigbreak

	\begin{addmargin}[1em]{}
		At a high level, Android and iOS are very similar in their goals to be
		mobile friendly, relatively lightweight (compared to a desktop os) operating systems.
		Where iOS was initially largely based off the core of the Mac OS X operating system and
		thus uses the XNU kernel of Darwin, Android was built off of the Linux kernel.
		Though different in many regards (XNU is hybrrid, Linux is monolithic), both are
		unix-like kernels, and aim to be POSIX compatible. \\ \bigbreak

		Overall, iOS is built explicitly to be run on Apple devices (iPhone \& iPad) where as
		Android, being completely open source (with exception of some drivers) in nature can
		run on nearly any device meeting minimum resource requirements. \\ \bigbreak

		On the subject of open vs closed source, iOS is largely closed source, with small
		bits of open source components at its core (XNU for example is open source) in contrast
		to Android. \\
	\end{addmargin}

%% Problem 6
\bigbreak
\textbf{6. (25 points)} Review the memory management for a C program (read the recitation
notes if you haven't got the chance to attend recitation), and then study the following C
code and answer questions:
\inputminted{c}{p6.c}
Where (data segment, stack, or heap) are
	(a) variable a,
	(b) variable b,
	(c) the space pointed by b,
	(d) variable c,
	(e) the space pointed by c, and
	(f) variable d
stored respectively? After foo finishes its execution,
which of the above variable/space are/is reclaimed by the OS?
\textcolor[RGB]{240,240,240}{\rule{\textwidth}{0.5pt}}\bigbreak

	\begin{addmargin}[1em]{}
		\textbf{(a)} Global variable $\Rightarrow$ \textbf{data segment}. This value
			will not be reclaimed after foo finishes executing and will exist
			for the entire duration of the programs execution.
		\bigbreak
		\textbf{(b)} Local variable $\Rightarrow$ \textbf{stack}. Exists only in the scope of
			foo, once foo is done executing, the variable will be popped off the stack
			(reclaimed by OS).
		\bigbreak
		\textbf{(c)} Malloc'd data $\Rightarrow$ \textbf{heap}. Variable 'b' points to malloc'd
			data in the heap, this will not be reclaimed by the OS until 'free' is 
			called on that address. As this never happens in our program, we actually
			have a memory leak here.
		\bigbreak
		\textbf{(d)} Local variable $\Rightarrow$ \textbf{stack}. Exists only in the scope of
			foo, once foo is done executing, the variable will be popped off the stack
			(reclaimed by OS).
		\bigbreak
		\textbf{(e)} Variable 'c' points to the address of 'a'. This means it is 
			stored in the \textbf{data segment}. And, just like 'a' (because it is a),
			it will exist for the entire duration of the programs execution. Because no
			'malloc' was used, no corresponding 'free' is necessary.

		\bigbreak
		\textbf{(f)} Local variable $\Rightarrow$ \textbf{stack}. Exists only in the scope of
			foo, once foo is done executing, the variable will be popped off the stack
			(reclaimed by OS).
	\end{addmargin}

%% Problem 7
\bigbreak
\textbf{7. (25 points)} When the following C program is run in a Linux system,
the execution of which line(s) must trigger invocation(s) of system call and/or exception?
When will the program terminate?
\inputminted{c}{p7.c}
\textcolor[RGB]{240,240,240}{\rule{\textwidth}{0.5pt}}\bigbreak

	\begin{addmargin}[1em]{}
	\end{addmargin}

\end{document}
